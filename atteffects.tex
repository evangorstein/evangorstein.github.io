\documentclass{article}
\usepackage[utf8]{inputenc}

\title{Blog Post on Attributable Effects}
\author{evangorstein }
\date{January 2021}

\begin{document}

\maketitle

\section{Introduction}



\section{Attributable Effects}

One limitation of each of the tests and estimates described above is that they model the treatment effect as constant across individuals. As a result, they restrict the set of possible additive treatment effects $\theta=(r_{T11}-r_{C11}, r_{T12}-r_{C12}, ... , r_{TI2}-r_{CI2})$ to a one-dimensional subspace of the parameter space $\mathbb{R}^{2I}$, wherein the effects do not differ between individuals. In practice, we know that there may be significant variability in the effect of a treatment across individuals, and we would like some way of getting a handle on $\theta$ without making any assumptions about its form at the outset. One way to do this is with the concept of ``attributable effects".



\section{Conclusion}

Something about how usually, we cannot afford to do a randomized experiment and we're compelled to rely instead on performing inference from observational studies. The method we present, which is valid for a randomized experiment, should provide the goal post--in other words, we should aim to design our observational studies so that this method is approximately valid. 



\end{document}
